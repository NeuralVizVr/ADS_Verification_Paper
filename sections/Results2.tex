\section{Results and Discussion}
\label{sec:results}

Building on the methodology of Sections~\ref{sec:nn-architectures}--\ref{sec:verificationApproach},
we systematically probe robustness across four orthogonal axes and two
independent verifiers.  
Table~\ref{tab:exp-grid} lists the levels in each dimension; their full
cross-product defines our evaluation matrix.

\begin{table}[h]
\centering
\caption{Experimental grid used throughout the results section.}
\label{tab:exp-grid}
\begin{tabular}{@{}lp{0.70\linewidth}@{}}
\toprule
\textbf{Factor} & \textbf{Levels} \\ \midrule
Model capacity & \textit{39}, \textit{116}, \textit{520}, \textit{924} parameters (CNNs; Section~\ref{sec:nn-architectures}). \\[2pt]
Training regime & Vanilla; PGD adversarial training with \(\varepsilon \in \{15,25\}/255\). \\[2pt]
Safety property & \(\mathbf{p_1}\), \(\mathbf{p_2}\), \(\mathbf{p_3}\), \(\mathbf{p_4}\) (Section~\ref{sec:formal-properties}). \\[2pt]
Perturbation radius & \(\ell_\infty\) bounds of \(3/255\), \(5/255\), \(10/255\), \(20/255\). \\[2pt]
Verifier & \(\alpha,\beta\)\textsc{-CROWN}  and \textsc{Marabou} . \\ \bottomrule
\end{tabular}
\end{table}

\paragraph{Baseline accuracy.}
The vanilla networks attain clean-test accuracies of
\(91.5\%\), \(97\%\), \(98.5\%\), and \(99\%\) for the 39, 116, 520, and 924 models,
respectively, on the 200-image test set introduced in
Section~\ref{sec:data_collection}. 

\paragraph{Verification workload.}
From this corpus we select
\textbf{100 images} that every vanilla model classifies correctly.
Crossing the factors of Table~\ref{tab:exp-grid} with these 100 inputs yields
\[
4 \times 3 \times 4 \times 4 \times 2 \times 100 \;=\; 38\,400
\]
individual verification queries.  
The remainder of Section~\ref{sec:results} dissects their outcomes along
five lenses: model capacity, training regime, verifier behaviour, property
difficulty, and perturbation radius.