\section{Related Work} \label{sec:related_work}

Formal verification of neural networks (NNs) has become an increasingly critical research area, particularly as NNs are deployed in safety-critical systems such as autonomous vehicles \cite{katz2017reluplex, liu2021algorithms}. The goal is to provide mathematical guarantees about a network's behavior, moving beyond empirical testing which may miss rare but safety-critical corner cases \cite{huang2020survey}. Autonomous driving systems (ADS) present a unique challenge due to high-dimensional sensory inputs (e.g., camera images, LiDAR), dynamic operational environments, and the severe consequences of failure \cite{tran2020verification, dreossi2019verifai}.

\paragraph{Verification Approaches.} Methods for NN verification can be broadly categorized into \emph{complete} and \emph{incomplete} techniques. Complete verifiers, such as those based on Satisfiability Modulo Theories (SMT) or Mixed Integer Linear Programming (MILP), aim to definitively prove or disprove a property. Tools like Reluplex \cite{katz2017reluplex} and Marabou \cite{katz2019marabou} belong to this category, extending solvers with custom rules for handling piecewise-linear activations. While these methods can provide exact results, their scalability remains limited for large or deep architectures \cite{gehr2018ai2, ehlers2017formal}.

Incomplete methods trade absolute guarantees for improved scalability, using over-approximation to certify properties. Examples include abstract interpretation \cite{singh2019abstract, eran}, reachability analysis \cite{huang2019reachnn}, and bound propagation techniques \cite{zhang2018efficient, xu2021fast, wang2021beta}. \(\alpha,\beta\)-CROWN \cite{wang2021beta} is a state-of-the-art incomplete verifier that combines optimized linear bound propagation with branch-and-bound search, excelling in \(L_\infty\) robustness certification.

\paragraph{SAT-based Verification.} A third category, exemplified by NeuralSAT, formulates verification as a Boolean satisfiability (SAT) problem by encoding network constraints into CNF. SAT-based approaches can provide completeness for certain property classes while benefiting from decades of SAT solver optimization, though they often struggle with scaling to very large networks.

\paragraph{Verification in ADS.} Several studies have targeted ADS-specific verification. Sun et al. \cite{sun2019formal} verified safety properties for NN-controlled robots navigating polyhedral obstacles, emphasizing perception modeling. Habeeb et al. \cite{habeeb2023verification} proposed a falsification-driven approach to identify unsafe trajectories in camera-based ADS. Ivanov et al. \cite{ivanov2020case} developed a benchmark for verifying NN controllers in autonomous racing, revealing challenges in bridging the simulation-to-reality (sim2real) gap. Dutta et al. \cite{dutta2019reachability} applied reachability analysis to continuous dynamics in NN-controlled systems.

\paragraph{Benchmarking Initiatives.} The VNN-COMP competition \cite{bak2021second, brix2023fourth} has significantly shaped the field by providing standardized benchmarks and fostering rapid tool development. Tools such as \(\alpha,\beta\)-CROWN, Marabou, and NeuralSAT regularly participate, often ranking among top performers in their respective categories.

\paragraph{Prior Work and Gap.} Bukhari et al. \cite{bukhari2024creating}, including authors of the present study, reported on creating and training a formally verified NN for autonomous navigation, noting installation, compatibility, and suitability challenges across verifiers. However, there remains a lack of systematic, side-by-side evaluation of leading verification tools on the same ADS dataset, across multiple robustness properties and perturbation levels. This paper addresses that gap through a detailed benchmarking study of \(\alpha,\beta\)-CROWN, Marabou, and NeuralSAT in a controlled Unity-based ADS environment.
